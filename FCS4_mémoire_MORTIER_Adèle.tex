\documentclass[french]{article}
\usepackage[T1]{fontenc}
\usepackage[utf8]{inputenc}
\usepackage{float}
\usepackage{lmodern}
\usepackage[a4paper]{geometry}
\usepackage{csquotes}
\usepackage[style=authoryear,backend=biber]{biblatex}
\usepackage[french]{babel}
\usepackage{hyperref}
\addbibresource{biblio.bib}
\title{Mémoire et filiation chez l'homme}
\begin{document}
	\maketitle
	\section*{Introduction}
		Le concept de mémoire fait régulièrement l'objet de métaphores du type ``ce souvenir est gravé en moi'', ``il porte un bien lourd passé'', ``il a un héritage difficile'' etc. Comme si, d'une manière ou d'une autre, notre mémoire n'était pas simplement faite de nos souvenirs individuels, mais des souvenirs de nos aïeux, qui se seraient physiquement imprimés en nous, et ce, malgré nous. Cette intuition que nous avons sur la mémoire quand nous la mentionnons est-elle vérifiée? Peut-on réellement ``hériter'' des souvenirs d'un groupe, et en particulier des souvenirs de sa famille? Dans ce document, on se propose d'aborder le problème selon trois angles, trois échelles, trois moments. On étudiera en premier lieu les facteurs génétiques qui pourraient expliquer l'héritabilité de certains souvenirs, en particulier traumatiques. On verra, en second lieu, que le développement du fœtus est aussi un moment de partage privilégié avec la mère, favorisant la formation de souvenirs en commun. Enfin, on s'intéressera à la formation de souvenirs partagés au cours de la vie, par la médiation d'un contexte social et/ou historique lié à la famille. Dans une second partie, on comparera, on questionnera, et, éventuellement, on critiquera ces différentes approches.
	\section{Des facteurs à plusieurs niveaux}
		Nous pensons que la part héritable de la mémoire peut être divisée en trois grand types, qui introduisent une typologie en terme d'échelle (du cellulaire à l'inter-individuel), mais aussi de temporalité (d'avant la naissance à la vie d'adulte au sein d'une communauté donnée). 
		\subsection{Niveau cellulaire : une mémoire ``génétique''}
			Dans cette section, on s'intéresse à la transmission d'une mémoire au niveau génétique, ou plus précisément épigénétique.
			\subsubsection{Qu'est-ce que l'épigénétique}
				L'épigénétique s'oppose à la génétique en tant qu'elle ne concerne pas les modifications de la séquence d'ADN. L'épigénétique concerne les modifications dans l'expression des gènes (à quelle fréquence chaque gène ``ordonne'' la formation de la molécule qu'il code), elle agit donc à un second niveau. Mais ce niveau n'en est pas moins capital; car l'expression des gènes est déterminante pour le futur de la cellule. A ce titre, c'est l'expression des gènes qui explique la différence entre les différentes cellules du corps, mais aussi, qui détermine le sexe chez certains animaux.\\
				L'expression des gènes dépend de l'état de la molécule d'ADN, et de son degré d'enroulement autour de protéines histones; un gène totalement enroulé autour de son histone ne peut s'exprimer; un gène totalement déroulé peut s'exprimer. Le facteur principal de modification de l'ADN ou des histones est la méthylation (influence d'un groupement méthyle).\\
				Dans le cas de l'ADN, la méthylation entraîne l'addition d'un groupement méthyle à la chaîne, au niveau de sites CpG (cytosine-phosphate-guanine), qui sont les points de la chaîne où une cytosine est directement suivie par une guanine. En particulier, si cette méthylation touche le promoteur d'un gène (région qui conditionne la transcription d'un gène en ARN), alors le gène en aval ne pourra plus s'exprimer; il sera inhibé. Dans le cas des histones, l'action d'un groupe méthyle favorise l'enroulement de l'ADN autour des histones, et donc, inhibe l'expression des gènes.\\
				La méthylation de l'ADN et de la chromatine (ADN et histones) jouent un rôle prépondérant dans l'``empreinte parentale'', c'est-à-dire les cas d'asymétrie d'expression entre la copie des gènes issues de la mère et issues du père. En effet, il a été récemment montré que la méthylation se conserve par mitose, mais aussi et surtout par méiose et lors de la fécondation; ce qui signifie que les caractères épigénétiques sont susceptibles d'être retrouvés chez les descendants.
			\subsubsection{Épigénétique et mémoire}
				On a vu que la méthylation était un une caractéristique du génotype à la fois acquise et transmissible. Mais quel est le lien entre la méthylation et la mémoire ou l'histoire d'un individu? Des études récentes on montré que la méthylation pouvait être déclenchée par des facteurs environnementaux bien précis; en particulier, les situations de stress intense, ou des conditions de vie particulièrement difficiles (manque de nourriture, manque de sommeil). Or, tout porte à croire que ces situations sont corrélées avec des souvenirs très forts, des souvenirs traumatiques. Ainsi, les traumatismes subis par un individus au cours de sa vie pourraient laisser des marques chez ses enfants, sous forme épigénétique.
				
		
		\begin{itemize}
			\item certaines modifications liées au vécu d'un individu peuvent être passées à ses descendants
			\item ces modifications sont du domaines de l'épigénétique : elles concernent l'expression des gènes
			\item le phénotype d'un individu peut être modifié sans que son génome ``en propre'' (parties codantes) ne le soit
		\end{itemize}
		\subsection{Niveau développemental : une mémoire implicite}
			Il apparaît maintenant que les souvenirs et les sensations de la mère lors de sa grossesse peuvent être ``transmis'' à son bébé. Bien sûr, on ne parle pas ici de télépathie; la transmission des souvenirs peut s'opérer \textit{via} des perceptions partagées (en particulier, perceptions gustatives, olfactives, auditives), ou \textit{via} des conditions de vie partagées (stress, restrictions alimentaires, consommation d'éléments toxiques, maladie etc.)
			
%			Cette transmission peut se faire selon deux modes : au niveau biologique (médiation par le placenta), et au niveau perceptuel (perception par le bébé d'éléments externes).
		%	\subsubsection{Processus biologiques partagés}
				Dans le ventre de sa mère, le bébé est en effet capable de percevoir des stimuli en extérieurs et intérieurs (\cite{busnel2010}), et de les ``apprendre'' (\textit{fetal learning}). Par ``apprendre'', on entend que le fœtus est susceptible de modifier son comportement par rapport à un stimulus, en fonction de son expérience du stimulus (\cite{james2010}). On dira qu'un stimulus a été ``mémorisé'' par le fœtus si ce dernier a retenu  les comportements spécifiques au stimulus.  Ici, on se restreindra aux stimuli olfactivo-gustatifs et auditifs, et à l'apprentissage par exposition au stimulus (\textit{imprinting}).
				\begin{itemize}
					\item le système gustatif est fonctionnel entre 4 et 6 mois\footnote{à cet âge, le fœtus déglutit déjà plus souvent dans un milieu sucré};
					\item le système olfactif se développe vers 7 et 9 mois\footnote{le bébé est d'abord \textit{sensible} aux odeurs avant de pouvoir les discriminer clairement};
					\item le système auditif est mature vers 5 mois et demi, et entraîne des réponses à des stimuli à partir de 6 mois et demi\footnote{à cet âge, le fœtus connaît des accélérations de son rythme cardiaque lorsque des stimuli inconnus ou ``appréciés'' lui parviennent};
				\end{itemize}
			
			\subsection{Mémoire des goût et des odeurs}
				Les stimuli olfactivo-gustatifs en particulier, peuvent être mémorisés \textit{in utero} par le bébé si les saveurs qui leurs sont associées se retrouvent dans le liquide amniotique. Peu après la naissance, les saveurs peuvent également se retrouver dans le lait maternel. Ces stimuli ont par la suite de grandes chances d'être préférés à d'autres stimuli inconnus. C'est le cas d'aliments comme l'ail, le cumin, l'anis \cite{schaal2000}\footnote{\cite{schaal2000} montrent que les nouveaux-nés dont la mère a consommé des aliments à base d'anis durant sa grossesse (sirop, pastilles, cookies) n'expriment pas plus de dégoût envers l'arôme d'anis qu'envers un arôme contrôle, au contraire des enfants n'ayant pas été mis au contact de l'anis \textit{in utero} (ceux-ci expriment plus fréquemment du dégoût vis-à-vis de cette odeur que vis-à-vis du contrôle). De plus, les bébés ayant été mis au contact de l'anis \textit{in utero} présentent des mouvements de la bouche (signe de contentement) plus longtemps vis-à-vis de l'anis que vis-à-vis du contrôle, contrairement aux enfants n'ayant pas été mis au contact de l'anis précédemment.} ou la carotte\cite{mennella2001}\footnote{\cite{mennella2001} montrent que les bébés ayant été mis en contact avec de la carotte \textit{in utero via} le liquide amniotique ou bien après la naissance \textit{via} l'allaitement manifestent moins d'expressions de dégoût quand ils sont confrontés pour la première fois à de la nourriture solide à base de carotte; par ailleurs, leur niveau de contentement est mieux noté par la mère que dans les cas contrôles.}. La préférence peut durer plusieurs années (4 ans). Ainsi, les souvenirs gustatifs et par conséquent les goûts de la mère peuvent être passés à son enfant.
				Liquide amniotique qui fait passer les odeurs et les gouts
				continuité trans natale\\
			\subsection{Mémoire des sons}
				Concernant les stimuli auditifs, les expériences menées sont très nombreuses, mais, par rapport aux expériences impliquant des stimuli olfactivo-gustatifs, beaucoup demeurent limitées d'un point de vue méthodologique \cite{james2010}:
				\begin{itemize}
					\item il est difficile de contrôler les conditions (faire en sorte que le stimulus soit inédit, trouver de stimuli contrôle);
					\item  il est difficile de s'assurer que le stimulus cible n'ait pas été appris après la naissance.
				\end{itemize}
				Cela, il a été montré à plusieurs reprise les faits suivants:
				\begin{itemize}
					\item le nouveau né préfère les voix de femmes aux voix d'hommes;
					\item entre les voix d'hommes, il n'a pas de préférence pour la voix du père;
					\item entre les voix de femmes en revanche, il a une préférence pour la voix de la mère.
				\end{itemize}
				En conséquence, c'est surtout l'apprentissage de la voix de la mère qui est testé. La voix de la mère est un stimulus auditif de bonne qualité pour le fœtus, car il lui parvient de façon interne et externe.\\
				
				Une des études les plus sérieuses sur le sujet est \cite{deCasper1986}. Un passage de texte préenregistré était joué deux fois par jour par la mère durant les six dernières semaines de grossesse. Après la naissance, les préférences du nouveau né ont été testées par rapport à un autre passage inconnu. Le proxy indiquant la préférence du bébé était la fréquence de succion d'un mamelon artificiel (paradigme HAS)\footnote{normalisée par la fréquence \textit{baseline} du nouveau né}. Les préférences ont été comparées à celle de bébés contrôles, n'ayant pas été exposé aux passages \textit{in utero}, et celle de bébés exposés à des passages non lus par leur propre mère lors du test.\\
				Les auteurs ont montré que les passages connus avaient un pouvoir de renforcement systématiquement plus grand (il augmentait davantage la fréquence de succion) qui les passages inconnus. En particulier, ce pouvoir ne dépendait pas de la voix utilisée lors du test (celle de la mère ou d'une autre femme). Cela prouve que le renforcement est du au passage lui-même (caractéristiques linguistiques) et non au médium (la voix). Les bébés contrôles ne montraient pas de différence de comportement entre les deux passages.\\
				
				Des récents follow-up de cette étude tentent de s'affranchir encore davantage du biais lié à l'apprentissage post-natal, en faisant des mesures d'IRMf directement pendant la grossesse \cite{jardri2012}. 
			\subsection{Bilan}
				Les souvenirs dont nous avons parlé dans cette partie sont des souvenirs d'assez bas niveau, de type sensoriel. Ils sont appris de façon inconsciente par le fœtus, et remémorés de façon également inconsciente.
				De plus, 
		\begin{itemize}
			\item certains aspects du vécu de la mère peuvent être aussi ressentis par le fœtus \textit{in utero}, à mesure que les sens se développent
			\item ces ``souvenirs'' peuvent perdurer sous la forme de traces implicites après la naissance, et parfois des années plus tard
			\item par exemple : attirance pour certaines saveurs, stress vis à vis de certaines situations (?)
		\end{itemize}
		\subsection{Niveau socio-historique : une mémoire sémantisée et partagée}
			\subsubsection{Représentations de la mémoire familiale: un court détour par la littérature}
			La ``mémoire familiale'' est un \textit{topoï} répandu dans la littérature, et en particulier dans le roman. On pense bien entendu à l'atavisme zolien, à l'œuvre de Proust, mais surtout dans notre cas à \textit{W ou le souvenir d'enfance} de George Perec, ou \textit{La Route des Flandres} de Claude Simon.\\
			Dans \textit{W}, l'auteur mêle en effet un récit apparemment enfantin -- celui de l'île W -- à des bribes de souvenirs autobiographiques, relatant en particulier la séparation entre l'auteur et sa mère. Ces souvenirs personnels, comme le dit lui-même l'auteur, demeurent très sporadiques:
			\begin{center}
				\begin{minipage}{.7\textwidth}
				``Je n'ai pas de souvenirs d'enfance'': je posais cette affirmation avec assurance, avec presque une sorte de défi. L'on n'avait pas à m'interroger sur cette question. Elle n'était pas inscrite à mon programme. J'en étais dispensé: une autre histoire, la Grande, l'Histoire avec sa grande hache, avait déjà répondu à ma place: la guerre, les camps. A treize ans, j'inventai et dessinai une histoire. Plus tard, je l'oubliai. Il y a sept ans, un soir, à Venise, je me souvins tout à coup que cette histoire s'appelai ``W'' et qu'elle était, d'une certain façon, sinon l'histoire, du moins une histoire de moment de mon enfance.
			\end{minipage}
			\end{center}
			Mais est-il bien vrai que l'auteur n'a pas de souvenir, quand bien même les horreurs vécues par sa famille transparaissent clairement -- quoique de façon maquillée -- dans le récit de l'île W? Dans cette section, nous nous intéresseront à ces traces -- bonne ou mauvaises -- que laisse le passé d'une famille sur ses membres. \\
			
			La mémoire collective a pour la première fois été mise en avant par Halbwachs \cite{halbwachs1925,halbwachs1950}? En particulier, Halbwachs étudie la mémoire familiale dans le chapitre V de \cite{halbwachs1925}. Auparavant, la mémoire était avant tout appréhendée sur le plan individuel, et l'on ne concevait pas que la mémoire d'une personne donnée puisse être influencée par les souvenirs de personnes extérieures. La mémoire familiale se présente comme un niveau intermédiaire entre la mémoire culturelle au sens très large (mémoire nationale, voire transnationale...), et la mémoire autobiographique individuelle \cite{boesen2012}.\\
			
			\cite{muxel2012a,boesen2012,garstenauer2012} montrent ainsi que des relations de contagion existent à l'échelle familiale -- et qu'elles y sont particulièrement prégnantes. Il est montré notamment que la mémoire individuelle est sujette à un \textit{tradeoff} permanent avec la mémoire véhiculée par la famille dans son ensemble; et qu'elle doit donc faire des compromis pour satisfaire un critère d'homogénéité. La mémoire familiale joue également un très grand rôle dans la perception de soi, et dans la construction d'une identité à la fois sociale et personnelle.\\
			La mémoire familiale peu se définir par ses formes et ses fonctions. \cite{muxel2012a} insiste ainsi sur les différents degré de mémoire, et l'inscription de la mémoire familiale dans un jeu de ``poupées russes'':
			\begin{figure}[H]
				\centering
				\fbox{
					\begin{minipage}{.5\textwidth}
						mémoire familiale (commune, cohérente)\\
						~\\
						\fbox{
							\begin{minipage}{.8\textwidth}
								traces d'évènements (arbitraire, médiation directe)\\
								~\\
								\fbox{
									\begin{minipage}{.8\textwidth}
										mémoire physique (apparences, comportements)\\
										~\\
										\fbox{
											\begin{minipage}{.8\textwidth}
												mémoire sensorielle (intime, \textit{involontaire})\\
												~\\
										\end{minipage}}
									\end{minipage}
								}
							\end{minipage}
						}
					\end{minipage}
				}
				\caption{Différents niveaux de mémoire selon \cite{muxel2012a}}
			\end{figure}
			\begin{itemize}
				\item mémoire sensorielle: mémoire très intime, peu contrôlée (mémoire involontaire proustienne), très peu sémantisée;
				\item mémoire physique : mémoire des corps et des comportements peu sémantisée;
				\item traces d'évènements: mémoire fixée sur des supports (photographies, films...), commune, arbitraire, directement partagée;
				\item mémoire familiale: mémoire sémantisée (histoires, mythes...), très largement partagée (genre de ``common ground'' mémoriel), avec une forte cohérence.
			\end{itemize}
			Bien sûr, ces dimensions ne manquent pas de tisser des liens entre elles. La mémoire familiale ravive et modifie certains souvenirs personnels. Elle en inhibe d'autres.\\
			Les fonction de la mémoire familiale sont au nombre de trois:
			\begin{itemize}
				\item transmission
				\item reviviscence
				\item réflexivité
			\end{itemize}
			La fonction de transmission permet au groupe familial de s'éprouver identique à lui-même dans la succession du temps, tout en assurant le sentiment d'appartenance des individus au groupe. En ce sens, la notion de groupe se rapproche de celle d'individu : la famille est une entite supra-personnelle \cite{boesen2012}. Comme chez l'individu, les souvenirs qui ne sont pas cohérents avec l'identité familiale sont oubliés; ceux qui renforcent cette identité sont ressassés. Mais les souvenirs conservés par le groupe ne peuvent pas non plus contredire totalement les souvenirs des individus. Parmi les souvenirs transmis, on compte notamment les souvenirs liés aux origines du groupe, les souvenirs liés aux normes (sociales comportementales) que le groupe s'est donné, les souvenirs liés aux éléments marquants vécus par le groupe.\\
			La fonction de reviviscence est ancrée dans le présent; elle implique la résurgence puissante et inattendue de souvenirs familiaux (souvenirs d'enfance...). Cette fonction comporte une composante épisodique (contenu) et sensorielle (déclencheurs).\\
			La fonction réflexive est prospective: elle est tournée vers le futur. Elle consiste en une remise à plat du passé (d'où est-ce que je viens?) afin de préparer le futur (où est-ce que je vais?). Il s'agit de faire le bilan. Les processus contribuant à cette fonction sont aussi liés à la théorie de l'esprit, dans la mesure où il faut se représenter, soi et les autres, dans le passé pour mieux se projeter dans l'avenir. \\
			
			Enfin, la mémoire familiale laisse une place importante à l'oubli. L'oubli est en effet nécessaire dans la phase de ``négociation'' des souvenirs. L'oubli permet la cohérence. Mais le compromis et l'oubli posent un problème plus vaste, qui est celui de la notion de vérité inhérente aux faits. Faut-il conserver ses souvenirs tels quels car ils nous semblent plus ``vrais'', ou les troquer pour des mythes familiaux, qui coïncident peut-être moins avec la réalité, mais qui sont peut-être plus facile à porter? Pour ce qui est de la transmission, l'oubli permet la continuité mais aussi l'introduction de nouveaux souvenirs. Pour ce qui est de la reviviscence individuelle, l'oubli peut être salvateur en tant qu'il occulte les souvenirs douloureux. Pour ce qui est de la réflexivité, l'oubli permet d'approcher la vérité.
		\subsection{Famille et travail, rupture et continuité}
			\cite{boesen2012} s'intéresse à un cas particulier de mémoire familiale: celle des familles de fermiers au Luxembourg. Cette mémoire est intrinsèquement liée à un contexte social, mais surtout professionnel. Les familles de fermiers au Luxembourg s'articulent en effet autour de l'exploitation agricole, qui est aussi un espace de vie -- si bien qu'au lieu d'évoquer la famille, il vaudrait mieux parler de ``maison''. Cet attachement au travail est à mettre en perspective avec une notion particulière d'héritage (matériel, mais aussi ``spirituel'') chez ces familles. A cet égard, \cite{boesen2012} évoque la tradition du \textit{Bäisaz}, neveu du fermier sans enfant devenu son ``héritier''. Dans les familles rurales et agricoles, la continuité de l'exploitation et de la mémoire qu'elle véhicule prime donc sur les liens filiaux, et la pure ``génétique'' dont on a parlé dans les sections précédentes.\\
			
			Au travers d'une série d'interviews faisant intervenir 3 générations de différentes familles (comportant des \textit{Baïsaz}), \cite{boesen2012} met en avant l'attachement des ``grands-pères'' (individus de la première génération) pour l'exploitation familiale, et leur relative soumission aux impératifs qui leurs ont été assignés dans leur jeunesse. Au contraire, certains individus de la deuxième génération attribuent plus d'importance aux notions d'indépendance, d'individualité et de vie privée. La mémoire qu'ils revendiquent n'est donc plus simplement familiale, elle est aussi autobiographique. Il sont ainsi plus partagés:
			\begin{itemize}
				\item d'un côté, la famille et ses contraintes sont un poids: les jeunes sont fortement incités à devenir eux-même fermiers, pour garantir la cohérence et la continuité de la ``maison''; cela peut conduire à l'arrêt des études en vue de reprendre l'exploitation du père;
				\item d'un autre côté, la famille est perçue comme un moteur du succès; elle assure la cohésion des individus travaillant ensemble, qui partagent les mêmes principes, et sont liés par des liens de confiance forts.
			\end{itemize}
			Une tension existe donc entre la répugnance pour un destin ``tout tracé'', et les avantages liés à un héritage familial, à la fois matériel et immatériel. Cette tension peut se résoudre au moins partiellement par la mise en place d'une annexe à l'exploitation principale: c'est la fondation d'une nouvelle ``maison'' au sens propre comme au sens figuré. Ainsi, son gestionnaire peut s'émanciper, tout en restant dans la continuité des activités de la famille et en conservant des liens avec ses différents membres.\\
			
			Par ailleurs, les individus interviewé n'ont pas manqué de remarquer que le monde avait changé autour d'eux, en particulier en matière d'exigences professionnelles (diplômes et qualifications nécessaires pour exercer le métier d'exploitant agricole). C'est pourquoi, à la troisième génération surtout, la transmission de l'héritage familial et l'intégration au groupe ``nécessitent une forme de légitimation''. La transmission ne se fait plus par défaut, elle doit être méritée et être le fruit  d'une certaine ``détermination''
		
		
			
			
			
		\begin{itemize}
			\item l'enfant hérite parfois des souvenirs de ses parents, par la verbalisation et éventuellement la répétition de ceux-ci
			\item cet héritage peut perdurer de façon réellement transgénérationnelle, avec éventuellement des déformations
			\item ``mythes familiaux''
			\item famille dans ``le grand bain de l'histoire'' (héritage de l'esclavage, de la Shoah...)
		\end{itemize}
	\section{Des liens entre ces trois dimensions}
		\subsection{Exemple : le traumatisme}
			Le cas des souvenirs traumatiques illustre bien l'interconnexion qui peut exister entre les trois niveaux d'héritabilité de la mémoire que nous avons évoqués dans la première section.\\
			
			Au niveau épigénétique, il a été montré que les situations à fort potentiel traumatique, telles les situations de stress intense, des conditions de vie très difficiles lorsque la nourriture manque, étaient susceptibles de provoquer des modifications au niveau de l'ADN. Ces modifications, qui ne sont pas des mutations, n'affectent pas les parties codantes de l'ADN, elles affectent l'épigénome. Par exemple, \cite{vukojevic2014} se sont intéressés aux modifications par méthylation du promoteur du gène NR3C1 chez des survivants du génocide au Rwanda (plus précisément, l'ilôt CpG3). Le gène NR3C1 est en effet responsable de la synthèse des récepteurs aux glucocorticoïdes, des hormones impliquées dans la régulation de la reviviscence des souvenirs. Or, les mécanismes de la reviviscence jouent un rôle dans la survenue de souvenirs intrusifs, caractéristiques du trouble de stress post-traumatique (PTSD). \cite{vukojevic2014} ont découvert ont comparé les niveaux de méthylation entre des survivants du génocide atteints de PTSD plus ou moins fort, et des individus sains testé sur des tâches de mémorisation d'images. Ils ont découvert que le niveau de méthylation aux niveau du site CpG3 était négativement corrélé avec la reviviscence d'évènements traumatiques chez les hommes.\\
			
			Au niveau développemental, \cite{lee2014} s'est intéressé aux niveau socio-économique et à la santé d'individus ayant été exposé à la guerre de Corée \textit{in utero}. De nombreux facteurs sont mesurés grâce à des données de recensement. Il montre que les individus nés en 1950-51 ont en moyenne une moins bonnes santé, un niveau scolaire moindre, et un niveau de qualification moindre que les individus nés peu avant ou peu après (compte tenu de la tendance globale). Il remarque également que ces effets sont modulés par la zone géographique considérée (les zones centrales, plus fréquemment situées sur le front, sont plus affectées) et par la temporalité de la guerre (les individus nés en 1950 ont majoritairement des problèmes de santé, tandis que les individus nés en 1951, exposés à la guerre dans la phase critique de leur développement, ont un niveau socioéconomique plus bas).
			
			
		\begin{itemize}
			\item un traumatisme physique peut entraîner une modification de l'expression des gènes
			\item il peut aussi se ``transmettre'' au bébé et conditionner ses comportements futurs (cf. induction développementale...)
			\item il peut enfin affecter la famille toute entière (PTSD...)
		\end{itemize}
		\subsection{Différences et points de convergence}
%		cas du ptsd qui se transmet à l'enfant in utero : melange génétique épigénétique et developpement.
			On a pu voir que les traits hérités par l'épigénétique étaient caractérisés au niveau moléculaire et hormonal. C'est n'est donc pas une mémoire directe, mais une mémoire de très bas niveau, réflexe. Dans la plupart des cas, les individus ne sont pas conscients de causes de leurs symptômes; ils n'héritent pas des souvenirs épisodiques ou sémantiques de leurs parents, mais simplement des comportements corrélés ou causés par ces souvenirs. Le fait que ce type de ``mémoire'' soit réflexe la rend aussi très difficile à maîtriser et à contenir. \\
			
			Concernant les souvenirs ou les éléments appris \textit{in utero}, la mémoire impliquée est également de très bas niveau. Pour les souvenirs perceptuels (goûts, odeurs...), la mémoire a pour rôle d'assurer la continuité trans-natale : elle favorise les stimuli déjà rencontrés, sans autre justification\footnote{bien sûr, on pourrait trouver des explications évolutionnistes pour ce comportement}. Ainsi, le bébé ou le jeune enfant préférera certains aliments ou certaines voix, sera conscient de ces préférences, mais ne pourra pas nécessairement expliquer le pourquoi de ces préférences, autrement dit, il ne pourra pas élucider la source du souvenir. \\
			
			
		\begin{itemize}
			\item La ``mémoire'' portée par les gènes est une mémoire d'assez bas niveau, inconsciente, réflexe
			\item la mémoire familiale quant à elle est très abstraite
			\item toutes ces mémoires sont sujettes à des modifications lors de leur transmission : processus non déterministe
		\end{itemize}
		\subsection{Critique}
		\begin{itemize}
			\item Où s'arrête la mémoire? Existe-t-il vraiment une mémoire du corps? Le lien n'est-il pas trop ténu?
			\item Peut-on qualifier de souvenir des choses dont on ne sait pas explicitement qu'elles en sont? Quelles différence entre mémoire et vécu?
			\item Peut-on sortir de la mémoire familiale? Nous conditionne-t-elle totalement?
			\item Dans quelle mesure les dimensions interagissent-elles?
			\item Quels leviers?
		\end{itemize}
	\newpage
	\defbibheading{main}{\textbf{Supports principaux}}
	\defbibheading{minor}{\textbf{Supports secondaires}}
	\section*{Références}
	\printbibliography[heading=main, keyword=main]
	\printbibliography[heading=minor, keyword=minor]
	\nocite{*}

\end{document}
