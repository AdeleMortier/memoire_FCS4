\documentclass[french]{article}
\usepackage[T1]{fontenc}
\usepackage[utf8]{inputenc}
\usepackage{geometry}
\geometry{left=20mm, right=20mm, top=20mm, bottom=25mm}
\usepackage{float}
\usepackage{lmodern}
\usepackage{graphicx}
\usepackage{csquotes}
\usepackage[style=authoryear,backend=biber]{biblatex}
\usepackage[french]{babel}
\usepackage{hyperref}
\addbibresource{biblio.bib}
\title{Mémoire et filiation chez l'homme}
\begin{document}
	
	\maketitle
	\tableofcontents
	\section*{Introduction}
		Le concept de mémoire fait régulièrement l'objet de métaphores du type ``ce souvenir est gravé en moi'', ``il porte un bien lourd passé'', ``il a un héritage difficile'' etc. Comme si, d'une manière ou d'une autre, notre mémoire n'était pas simplement faite de nos souvenirs individuels, mais des souvenirs de nos aïeux, qui se seraient physiquement imprimés en nous, et ce, malgré nous. Cette intuition que nous avons sur la mémoire quand nous la mentionnons est-elle vérifiée? Peut-on réellement ``hériter'' des souvenirs d'un groupe, et en particulier des souvenirs de sa famille? Dans ce document, on se propose d'aborder le problème selon trois angles, trois échelles, trois moments. On étudiera en premier lieu les facteurs génétiques qui pourraient expliquer l'héritabilité de certains souvenirs, en particulier traumatiques. On verra, en second lieu, que le développement du fœtus est aussi un moment de partage privilégié avec la mère, favorisant la formation de ``souvenirs'' en commun \footnote{dans un sens que nous préciserons}. On s'intéressera ensuite à la formation de souvenirs partagés au cours de la vie, par la médiation d'un contexte social et/ou historique lié à la famille. Enfin, on appliquera ces différentes approches à un cas, celui de la mémoire de l'Holocauste. On conclura en comparant, en questionnant, et, éventuellement, en critiquant ces différentes approches.
		\section{Niveau cellulaire : une mémoire ``génétique''}\label{genetique}
			Dans cette section, on s'intéresse à la transmission d'une mémoire au niveau génétique, ou plus précisément épigénétique.
			\subsection{Qu'est-ce que l'épigénétique}
				L'épigénétique s'oppose à la génétique en tant qu'elle ne concerne pas les modifications de la séquence d'ADN\footnote{au niveau nucléotidique}. L'épigénétique concerne les \textbf{modifications dans l'expression des gènes}. Elle détermine à quelle fréquence chaque gène ``ordonne'', par l'intermédiaire de l'ARN, la formation de la molécule qu'il code. L'épigénétique a donc un rôle capital; car l'expression des gènes est déterminante pour le futur de la cellule. À ce titre, c'est l'expression des gènes qui explique la différenciation des cellules du corps, mais aussi, qui détermine le sexe chez certains animaux.\\
				
				L'expression des gènes dépend:
				\begin{itemize}
					\item de l'état de la molécule d'ADN;
					\item de son degré d'enroulement autour de protéines histones -- l'ensemble formé par les histones et l'ADN étant appelé chromatine;
					\item de la présence d'ARN non codant dans le cytoplasme.
				\end{itemize}\vspace{3mm}

				Le facteur principal de modification de l'ADN ou des histones est la \textbf{méthylation} (influence d'un groupement méthyle).
				\begin{figure}[H]
					\centering
					\includegraphics[width=300px]{../images/chromatin.png}
					\caption{Schéma de chromatine (histones+ADN) non compactée (euchromatine) et compactée (hétérochromatine) \href{https://nanobiologyhonoursprogrammeblog.wordpress.com/2018/04/18/3d-chromatin-conformation-in-disease-pathology/}{source}}
				\end{figure}
				Dans le cas de l'ADN, la méthylation entraîne l'addition d'un groupement méthyle à la chaîne, au niveau de \textbf{sites CpG} (cytosine-phosphate-guanine), qui sont les points de la chaîne où une cytosine est directement suivie par une guanine. En particulier, si cette méthylation touche le promoteur d'un gène (région qui conditionne la transcription d'un gène en ARN), alors le gène en aval ne pourra plus s'exprimer; il sera inhibé. Le groupe méthyle peut aussi agir sur les acides aminés d'une protéine histone. Cela favorise l'enroulement de l'ADN autour des histones. Or, un gène totalement enroulé autour de son histone ne peut s'exprimer. La méthylation des histones inhibe donc l'expression des gènes\footnote{l'acétylation des histones au contraire, conduit l'ADN à se dérouler et permet d'augmenter l'expression des gènes}.\\
				\begin{figure}[H]
					\centering
					\includegraphics[width=200px]{../images/DNA_methylation.png}
					\caption{Effet de la méthylation d'une cytosine au niveau de la molécule d'ADN}
				\end{figure}
			
				Un dernier facteur déterminant l'expression des gènes et la présence de fragments d'ARN non codant dans le cytoplasme. Contrairement à la méthylation qui inhibe la transcription de l'ADN en ARN, les fragments d'ARNnc agissent après la transcription, et empêchent la synthèse de la protéine correspondante au sein du cytoplasme (\cite{mansuy2016}). 
			\subsection{Transmission de la mémoire épigénétique}
				Lorsqu'un évènement extérieur (par exemple un traumatisme) survient, ce ne sont pas simplement les cellules du cerveau qui sont affectées par les modifications épigénétiques; les cellules sexuelles peuvent également être touchées. Cependant les cellules sexuelles ont une structure un peu différentes des autres cellules: les histones y sont remplacés par des protamines, des protéines permettant une meilleures compacité (et donc une meilleure protection) de l'ADN. Les modifications des protamines sont relativement peu connues chez l'humain, mais par contre, la méthylation de l'ADN et l'ARNnc jouent le même rôle dans les cellules sexuelles que dans les autres cellules (\cite{mansuy2016}).\\
				
				Or, la méthylation se conserve par \textbf{mitose}, mais aussi et surtout par \textbf{méiose} et lors de la fécondation; ce qui signifie que les caractères épigénétiques sont susceptibles d'être retrouvés chez les descendants. Des études visant à confirmer cette hypothèse ont été menées sur des modèles non humains comme les souris\footnote{ce qui permet un très grand contrôle des conditions expérimentales, et une échelle de temps réduite}. Un  ``traumatisme''\footnote{en l'occurrence, une séparation de la mère 3 heures par jour pendant les deux semaines suivant la naissance} subi par une souris se retrouve ainsi sur ses descendants à la troisième, voire à la quatrième génération:
				\begin{itemize}
					\item au niveau du phénotype: comportement dépressif et antisocial\footnote{abandon en situation de nage forcée; mauvaise mémoire (nature et position d'objets); retrait social ou problèmes de reconnaissance sociale...};
					\item au niveau de l'épigénome: taux de méthylation anormaux et comparables entre générations.
				\end{itemize}\vspace{2mm}
				Des transplantations d'ARNnc issu de gamètes de souris dépressives dans des œufs fertilisés de souris saines ont aussi donné lieu à des phénotypes dépressifs (\cite{mansuy2016}). Tout cela tend à prouver que la méthylation et l'ARNnc jouent un véritable rôle au niveau transgénérationnel et influent sur le phénotype des individus. Nous verrons plus loin (Section \ref{liens}) des études portant sur la méthylation chez l'humain.\\
				
				
				Cela dit, les marques épigénétiques peuvent aussi être acquises de façon \textbf{spontanée}, au cours du développement du fœtus. Ces modifications surviennent souvent lorsque l'environnement est défavorable: stress de la mère (\cite{vaiserman2017}), manque de nourriture (``cas d'école'' de la famine hollandaise \cite{heijmans2008}). Dès qu'elles sont déclarées, ces modifications sont stables d'une division cellulaire à l'autre, et pourront être transmises aux descendants. Les modifications interviennent surtout au moment de la gamétogenèse et de l'embryogenèse, quand l'épigénome est particulièrement sensible aux facteurs environnementaux. Mais la plasticité se prolonge au-delà même de la naissance, jusqu'au sevrage de l'enfant. Ainsi, un bébé peut développer des marques épigénétiques liées à un évènement dont ils ne rappellera pas formellement.
	
			\subsection{Bilan}
				L'épigénétique est un mécanisme susceptible de transmettre les stigmates d'un évènement (souvent délétère) d'un parent à son enfant. Les facteurs épigénétiques sont aussi très liés aux facteurs développementaux que nous allons étudier dans la section suivante. En effet, les modifications épigénétiques peuvent être déclenchées durant le développement du fœtus sous l'effet de l'environnement, sans être directement héritées. Cependant, il faut bien être conscient que la ``mémoire épigénétique'' est une mémoire d'extrêmement bas niveau, implicite. Deux individus partageant cette ``mémoire'' ne partageront ni leurs souvenirs épisodiques, ni leurs souvenirs sémantiques, ni leurs rêves, etc. Ils ne partageront qu'un ensemble de prédispositions pour certaines affections pouvant faire intervenir la mémoire. La relation est donc assez \textbf{ténue}. Nous reviendrons plus amplement sur le rôle de l'épigénétique dans le contrôle inhibiteur de la mémoire à la Section \ref{liens}.
				
		\section{Niveau développemental : une mémoire implicite}\label{develo}
			Il apparaît maintenant que les souvenirs et les sensations de la mère lors de sa grossesse peuvent être ``transmis'' à son bébé. La transmission des souvenirs peut s'opérer \textit{via} des perceptions partagées, ou plus largement \textit{via} des conditions de vie partagées.
			\subsection{Mémoire et perception}\label{perception}
				Dans le ventre de sa mère, le bébé est capable de \textbf{percevoir} des stimuli extérieurs et intérieurs (\cite{busnel2010}), et de les ``apprendre'' (\textbf{\textit{fetal learning}}). Par ``apprendre'', on entend que le fœtus est susceptible de modifier son comportement par rapport à un stimulus, en fonction de son expérience du stimulus (\cite{james2010}). On dira qu'un stimulus a été ``mémorisé'' par le fœtus si ce dernier a retenu  les comportements spécifiques au stimulus.  Ici, on se restreindra aux stimuli olfactivo-gustatifs et auditifs, et à l'apprentissage par exposition au stimulus (\textit{imprinting}). À noter que:
				\begin{itemize}
					\item le système gustatif est fonctionnel entre 4 et 6 mois\footnote{à cet âge, le fœtus déglutit déjà plus souvent dans un milieu sucré};
					\item le système olfactif se développe vers 7 et 9 mois\footnote{le bébé est d'abord \textit{sensible} aux odeurs avant de pouvoir les discriminer clairement};
					\item le système auditif est mature vers 5 mois et demi, et entraîne des réponses à des stimuli à partir de 6 mois et demi\footnote{à cet âge, le fœtus connaît des accélérations de son rythme cardiaque lorsque des stimuli inconnus ou ``appréciés'' lui parviennent};
				\end{itemize}\vspace{2mm}
				On peut donc affirmer qu'au troisième trimestre de grossesse, le fœtus est globalement capable de percevoir et de réagir à des stimuli sensoriels. C'est donc à cette période qu'il est le plus susceptible d'apprendre et de se souvenir.
			
			\subsubsection{Mémoire des goût et des odeurs}
				Les stimuli olfactivo-gustatifs en particulier, peuvent être mémorisés \textit{in utero} par le bébé si les saveurs qui leurs sont associées se retrouvent dans le \textbf{liquide amniotique}. Peu après la naissance, les saveurs peuvent également se retrouver dans le \textbf{lait maternel}. Ces stimuli ont par la suite de grandes chances d'être préférés à d'autres stimuli inconnus. C'est le cas d'aliments comme l'ail, le cumin, l'anis ou la carotte.\\
				
				\cite{schaal2000} par exemple montrent que les nouveaux-nés dont la mère a consommé des aliments à base d'anis durant sa grossesse (sirop, pastilles, cookies) n'expriment pas plus de dégoût envers l'arôme d'anis qu'envers un arôme contrôle, au contraire des enfants n'ayant pas été mis au contact de l'anis \textit{in utero} (ceux-ci expriment plus fréquemment du dégoût vis-à-vis de cette odeur que vis-à-vis du contrôle). De plus, les bébés ayant été mis au contact de l'anis \textit{in utero} présentent des mouvements de la bouche (signe de contentement) plus longtemps vis-à-vis de l'anis que vis-à-vis du contrôle, contrairement aux enfants n'ayant pas été mis au contact de l'anis précédemment.\\
				
				Une autre étude (\cite{mennella2001}) montre que les bébés ayant été mis en contact avec de la carotte \textit{in utero via} le liquide amniotique ou bien après la naissance \textit{via} l'allaitement manifestent moins d'expressions de dégoût quand ils sont confrontés pour la première fois à de la nourriture solide à base de carotte; par ailleurs, leur niveau de contentement est mieux noté par la mère que dans les cas contrôles. La préférence peut durer plusieurs années (4 ans). Ainsi, les souvenirs gustatifs et par conséquent les goûts de la mère peuvent être passés à son enfant.\\
				
				Le liquide amniotique, qui transmet les odeurs et les goûts au fœtus, et plus tard le lait maternel, assurent donc une forme de ``\textbf{continuité trans-natale}'' des préférences.\\
			\subsubsection{Mémoire des sons}
				Concernant les stimuli auditifs, les expériences menées sont très nombreuses, mais, par rapport aux expériences impliquant des stimuli olfactivo-gustatifs, beaucoup demeurent limitées d'un point de vue méthodologique (\cite{james2010}):
				\begin{itemize}
					\item il est difficile de contrôler les conditions (faire en sorte que le stimulus soit inédit, trouver de stimuli contrôle);
					\item  il est difficile de s'assurer que le stimulus cible n'ait pas été appris après la naissance.
				\end{itemize}\vspace{2mm}
				Cela dit, il est désormais acquis que:
				\begin{itemize}
					\item le nouveau né préfère les voix de femmes aux voix d'hommes;
					\item entre les voix d'hommes, il n'a pas de préférence pour la voix du père;
					\item entre les voix de femmes en revanche, il a une préférence pour la voix de la mère.
				\end{itemize}\vspace{2mm}
				En conséquence, c'est surtout l'apprentissage de la \textbf{voix de la mère} qui est testé. La voix de la mère est un stimulus auditif de bonne qualité pour le fœtus, car il lui parvient de façon interne et externe.\\
				
				Une des études les plus sérieuses sur le sujet a été menée par \cite{deCasper1986}. Dans cette étude, un passage de texte préenregistré était joué deux fois par jour par la mère durant les six dernières semaines de grossesse. Après la naissance, les préférences du nouveau né ont été testées par rapport à un autre passage inconnu. Le proxy indiquant la préférence du bébé était la fréquence de succion d'un mamelon artificiel (paradigme HAS)\footnote{normalisée par la fréquence \textit{baseline} du nouveau né}. Les préférences ont été comparées à celles de bébés contrôles, n'ayant pas été exposé aux passages \textit{in utero}, et celles de bébés exposés à des passages non lus par leur propre mère lors du test.\\
				Les auteurs ont montré que chez les bébés testés (et non les bébés contrôles) les passages connus avaient un pouvoir de renforcement systématiquement plus grand \footnote{il augmentait davantage la fréquence de succion} que les passages inconnus. En particulier, ce pouvoir ne dépendait pas de la voix utilisée lors du test (celle de la mère ou d'une autre femme). Cela prouve que le renforcement était du au passage lui-même et non au médium : le bébé a donc été capable d'``apprendre'' des caractéristiques linguistiques\footnote{bien sûr, ces caractéristiques sont dans doute de très bas niveau : prosodie etc.} du texte lu.\\
				
				Des récents \textit{follow-up} de cette étude tentent de s'affranchir encore davantage du biais lié à l'apprentissage post-natal, en faisant des mesures d'IRMf \textit{in vivo} directement pendant la grossesse, durant troisième trimestre (\cite{jardri2012}). Mais à notre connaissance ce genre de protocole n'est pas encore utilisé sur de larges cohortes.
				
			\subsection{Mémoire et stress}\label{stress}
				Outre les perceptions, la mère et son bébé sont susceptibles de partager des émotions, en particulier le \textbf{stress}. On parle dans de cas de \textit{\textbf{fetal programming}}. Le stress est notamment régulé par l'axe hypothalamus-hypophyse-cortex surrénal (\textbf{HPA}). Cet axe assure notamment la réponse dite ``\textit{fight or flight}'', et la formation de souvenirs à long terme liés à l'évènement déclencheur du stress. Le produit final de ce mécanisme est le \textbf{cortisol}, l'``hormone du stress''. Le stress maternel, au travers du cortisol, peut alors:
				\begin{itemize}
					\item modifier l'expression des gènes du fœtus \footnote{ce qui établit un lien entre environnement durant le développement et épigénétique};
					\item modifier le niveau de stress du fœtus.
				\end{itemize} \vspace{3mm}
				\begin{figure}[H]
					\centering
					\includegraphics[width=200px]{../images/hpa_axis.png}
					\caption{Axe HPA; le cortisol agit par \textit{feedback} négatif sur l'hypothalamus et l'hypophyse pour réguler la réaction de stress}
					\label{fig:hpa}
				\end{figure}
				\cite{baibazarova2013} ont ainsi étudié de façon longitudinale les liens entre le stress de la mère et celui de l'enfant\footnote{les facteurs épigénétiques ne sont pas étudiés, même s'ils peuvent entrer en ligne de compte dans les résultats}. Ont été comparés le taux de cortisol (de la mère, du liquide amniotique), le stress subjectif ressenti par la mère, l'état de santé du nouveau-né (son poids, son âge gestationnel), et son tempérament ultérieur (à 3 mois, évalué par sa mère\footnote{ce qui n'est pas tout à fait objectif, car il pourrait exister une corrélation entre le stress de la mère et son interprétation du tempérament de l'enfant}). Dans le tempérament, la peur face à des situations inattendues ou nouvelles, ainsi que l'inconfort dans des situations de contrainte sont évalués. Un stress élevé pendant la grossesse, traduit par des niveaux de cortisol élevés, est susceptible d'affecter le développement physique, neuronal et cognitif de l'enfant. D'importantes corrélations ont de fait été trouvées:
				\begin{itemize}
					\item corrélation positive entre le taux de cortisol de la mère et du liquide amniotique;
					\item corrélation négative entre le taux de cortisol amniotique et le poids de l'enfant à la naissance, et l'âge gestationnel (ces deux derniers facteurs étant eux-mêmes positivement corrélés entre eux);
					\item corrélation négative entre le poids de l'enfant et son tempérament à trois mois (peur et inconfort, également positivement corrélés entre eux).
				\end{itemize}\vspace{3mm}
				Cela dit, aucune corrélation n'a été reportée entre le stress subjectif de la mère et les autres indicateurs. Autrement dit, c'est un stress inconscient qui peut être ``transmis'' à l'enfant. Par ailleurs, le taux de cortisol de la mère n'est pas directement corrélé aux indicateurs propres à l'enfant (poids, âge gestationnel, tempérament); mais il peut exercer une action indirecte.
				\begin{figure}[H]
					\centering
					\includegraphics[width=300px]{../images/stress_pathway.jpg}
					\caption{Corrélations entre taux de cortisol, santé du nouveau-né et comportements du bébé (\cite{baibazarova2013})}
				\end{figure}
			
			\subsection{Bilan}
				Les souvenirs dont nous avons parlé dans cette partie sont des souvenirs d'assez bas niveau, de type sensoriel et émotionnel. Ils sont appris de façon inconsciente par le fœtus, et remémorés de façon également inconsciente, voire réflexe. En particulier, l'enfant ne parviendra pas à élucider la \textit{\textbf{source}} du souvenir. Cela est bien sûr à mettre en relation avec le fait que le cerveau du fœtus et de l'enfant n'est pas assez développé pour garantir une conservation durable et une sémantisation\footnote{qui passe aussi par la verbalisation} des traces mémorielles (``amnésie infantile'').\\
				Les souvenirs fœtaux n'ont par ailleurs pas une durabilité très attestée. Celle-ci dépend du stimulus considéré. Mais peu d'études traitant des préférences sensorielles et du stress proposent un \textit{design} longitudinal censé valider la conservation des caractères acquis \textit{in utero} durant l'enfance et l'adolescence. Cette volatilité des souvenirs du fœtus pourrait constituer une limite de ce type de mémoire\footnote{ même si bien sûr, on peut imaginer qu'une comportement initié durant la petite-enfance puisse être renforcé par exposition répétée au même stimulus pendant l'enfance et l'adolescence. Cela pourrait provoquer un effet ``boule de neige'' -- un enrichissement et renforcement du souvenir initial par des souvenirs postérieurs}.
				
		\section{Niveau socio-historique : une mémoire sémantisée et partagée}\label{socio}
			Après la naissance, l'enfant arrive dans une famille avec sa propre histoire, ses propres habitudes. Ses parents et les autres membres de la famille ont pour rôle de lui transmettre toutes ces informations de façon implicite ou explicite. Dans cette section, nous nous intéresserons donc aux traces que laisse le passé d'une famille sur ses membres. L'angle adopté sera essentiellement sociologique.
		
		\subsection{La mémoire familiale comme mémoire collective}\label{collec}
			La ``\textbf{mémoire collective}'' a pour la première fois été théorisée par Maurice Halbwachs (\cite{halbwachs1925,halbwachs1950}). Auparavant, la mémoire était avant tout appréhendée sur le plan individuel, et l'on ne concevait pas que la mémoire d'une personne donnée puisse être influencée par les souvenirs de personnes extérieures.\\
			En particulier, Halbwachs étudie la mémoire familiale dans le chapitre V des \textit{Cadres sociaux de la mémoire}.  Cette mémoire familiale se présente comme un niveau intermédiaire entre la mémoire culturelle au sens très large (mémoire nationale, voire transnationale...), et la mémoire autobiographique individuelle (\cite{boesen2012}). La mémoire familiale est décrite comme un \textbf{réseau} de mémoires qui s'harmonisent entre elles afin de former un tout homogène. Les particularités individuelles sont estompées; la mémoire familiale s'avère en effet différente d'une ``suite de tableaux collectifs'' (\cite[p.~110]{halbwachs1925}). L'\textbf{homogénéisation} possède également une dimension normative; les individus s'écartant du \textit{cadre} posé par la  mémoire familiale sont rejetés ou ignorés par la famille. Intégrer la mémoire individuelle à la mémoire familiale nécessite donc de faire d'opérer un \textit{tradeoff} (\cite{muxel2012}) entre son propre vécu et l'interprétation familiale des évènements. On parle aussi à ce sujet d'un double principe de \textbf{cohérence} et de \textbf{correspondance}.\\
			Outre l'homogénéisation, la mémoire familiale fait aussi l'objet d'une \textbf{agrégation} des souvenirs, en particulier des souvenirs épisodiques. Ceux-ci sont composés pour ne former plus qu'un seul souvenir, hybride, prototypique et hautement sémantisé. C'est le cas notamment des ``rituels'' familiaux (fêtes, rassemblements...). La mémoire familiale n'est donc pas linéaire, elle procède par couches successives et allées-venues\footnote{cf. également la notion bergsonienne de durée, qui se rapproche de cette interprétation... \cite[p.~5]{bergson1908}}. Chaque rappel du souvenir contribue à l'enrichir, à le rendre encore plus spécial, voire mythique\footnote{``[les souvenirs] se sont grossis de tout ce qui précède, et ils sont déjà gros de tout ce qui suit. A mesure qu'on s'y reporte plus souvent, qu'on y réfléchit davantage, loin de se simplifier, ils concentrent en eux plus de réalité [...]'' \cite[p.~114]{halbwachs1925}}.

		\subsection{Formes et fonctions de la mémoire familiale}\label{formefonc}
			\subsubsection{Formes}
			La mémoire familiale peu se définir par ses formes et ses fonctions. \cite{muxel2012} insiste ainsi sur les différents degrés de mémoire, et l'inscription de la mémoire familiale dans un jeu de ``poupées russes'':
			\begin{itemize}
				\item la ``mémoire sensorielle'' est très intime, peu contrôlée, très peu sémantisée;
				\item la ``mémoire physique'' est une mémoire des corps et des comportements (cf. également \cite[p.~121]{halbwachs1925});
				\item les ``traces d'évènements'' constituent une mémoire fixée sur des supports (photographies, films...); elle est commune, arbitraire, directement partagée (sans médiation);
				\item la ``mémoire familiale'' à part entière est une mémoire sémantisée (histoires, mythes...), très largement partagée (genre de \textit{common ground} mémoriel), avec une forte cohérence.
			\end{itemize}
			\begin{figure}[H]
			\centering
			\fbox{
				\begin{minipage}{.5\textwidth}
					mémoire familiale\\
					~\\
					\fbox{
						\begin{minipage}{.8\textwidth}
							traces d'évènements\\
							~\\
							\fbox{
								\begin{minipage}{.8\textwidth}
									mémoire physique\\
									~\\
									\fbox{
										\begin{minipage}{.8\textwidth}
											mémoire sensorielle\\
											~\\
									\end{minipage}}
								\end{minipage}
							}
						\end{minipage}
					}
				\end{minipage}
			}
			\caption{Différents niveaux de mémoire selon \cite{muxel2012}}
		\end{figure}
	\subsubsection{Fonctions}
			Bien sûr, les dimensions que nous avons énumérées entretiennent des liens. La mémoire familiale ravive et modifie certains souvenirs personnels. Elle en inhibe d'autres. Plus précisément, la mémoire familiale aurait trois fonctions: \textbf{transmission}, \textbf{reviviscence} et \textbf{réflexivité}.\\
			
			La fonction de transmission permet au groupe familial de s'éprouver identique à lui-même dans la succession du temps, tout en garantissant le sentiment d'appartenance des individus à ce même groupe. Comme à l'échelle de l'individu, les souvenirs qui ne sont pas cohérents avec l'identité familiale sont oubliés; ceux qui renforcent cette identité sont ressassés (cf. également les théorie des ``schémas'', \cite{bartlett1933}). La fonction de transmission a donc un rôle de sémantisation. Parmi les souvenirs transmis, on compte notamment les souvenirs liés aux origines du groupe, les souvenirs liés aux normes (sociales, comportementales) que le groupe s'est donné (cf. également \cite[p.~110-111]{halbwachs1925}), les souvenirs liés aux éléments marquants vécus par le groupe (arrivée d'un nouveau membre \cite[p.~121]{halbwachs1925}, décès, évènement historique à plus large portée...).\\
			
			La fonction de reviviscence est ancrée dans le présent; elle implique la résurgence puissante et inattendue de souvenirs familiaux (souvenirs d'enfance...). Cette fonction comporte une composante épisodique (pour le contenu) et sensorielle (pour les ``déclencheurs'').\\
			
			La fonction réflexive est prospective: elle est tournée vers le futur. Elle consiste en une remise à plat du passé (d'où est-ce que je viens?) afin de préparer le futur (où est-ce que je vais?). Il s'agit de faire le bilan. Les processus contribuant à cette fonction sont aussi liés à la théorie de l'esprit, dans la mesure où il faut se représenter, soi et les autres, dans le passé pour mieux se projeter dans l'avenir (\cite{buckner2007}). \\
			
			Enfin, \cite{muxel2012} remarque que la mémoire familiale laisse une place importante à l'\textbf{oubli}. L'oubli est en effet nécessaire dans la phase de ``négociation'' des souvenirs. L'oubli permet la cohérence. Mais le compromis et l'oubli posent un problème plus vaste, qui est celui de la notion de vérité inhérente aux faits. Faut-il conserver ses souvenirs tels quels car ils nous semblent plus ``vrais'', ou les troquer pour des mythes familiaux, qui coïncident peut-être moins avec la réalité, mais qui sont peut-être plus facile à porter? Nous verrons dans section suivante qu'il n'est pas forcément facile de répondre à cette question.
			
	
		\subsection{Bilan}
			On constate que les schémas propres à la mémoire autobiographique de l'individu se retrouvent à l'échelle de la famille. La famille est capable de se souvenir et groupe, de sémantiser des épisodes vécus, de créer des mythes, des stéréotypes ou des histoires qui se transmettent comme un patrimoine, d'homogénéiser les souvenirs pour conserver une identité propre, d'oublier pour se protéger. On peut aussi remarquer que les souvenirs se transmettent et ``mutent'' comme le feraient les gènes. La mémoire familiale est donc complexe et va sans doute l'être encore plus dans les décennies à venir, compte tenu des modifications importantes opérant sur la notion même de famille : familles mono- ou homo- parentales, familles recomposées, diasporas ``interconnectées''\footnote{on pense notamment à la communication par messagerie et par les réseaux sociaux, qui semble se développer au sein de la famille entre ses membres éloignés -- l'éloignement pouvant être générationnel ou spatial.}... Aussi, la conception traditionnelle de la parenté, qui se recoupait avec celle de génétique (``liens de sang''), sera sans doute de moins en moins répandue.
		
	\section{Des liens entre ces trois dimensions}\label{liens}
		\subsection{Un exemple : le traumatisme}
			Le cas des souvenirs traumatiques illustre bien l'interconnexion qui peut exister entre les trois niveaux d'héritabilité de la mémoire que nous avons évoqués dans la première section. Dans cette sous-section, nous traiterons le cas de la mémoire de l'Holocauste.
			\subsubsection{Le traumatisme et ses manifestations} 
				 Le traumatisme psychique est un dommage d'ordre psychologique et physiologique qui résulte d'un événement dramatiquement subi. La pathologie la plus associée au traumatisme est le trouble de stress post-traumatique (\textbf{PTSD}). Un patient atteint de PTSD sera sujet à des intrusions incontrôlables, suscitées par des éléments anodins soudainement associés à l'événement premier. Les intrusions peuvent être particulièrement violentes et le patient va alors ``revivre'' l'évènement traumatique. Certains patients atteints de PTSD présentent des dysfonctionnements dans l'axe HPA régulant la réponse au stress (cf. Figure \ref{fig:hpa}). Ils présenteront alors un taux étonnamment bas de cortisol, et auront des difficultés à former des souvenirs (\cite{vukojevic2014}). Nous allons voir que des marqueurs du PTSD peuvent se retrouver à différents niveaux chez les victimes de traumatismes.
				
			\subsubsection{Au niveau génétique}
				Le traumatisme de l'Holocauste et le stress produit peuvent se retrouver au niveau transgénérationnel dans l'épigénome de descendants de survivants. Dans une étude très récente, \cite{yehuda2016} se sont intéressés au gène \textbf{FKBP5}. Ce gène code en effet pour la protéine de même nom, qui interagit avec les \textbf{récepteurs aux glucocorticoïdes}\footnote{le cortisol est un glucocorticoïde}. Il joue donc un rôle indirect dans la régulation de l'axe HPA (cf. Figure \ref{fig:hpa}). En particulier, la méthylation des ilôts CpG du gène FKBP5 a été mesurée sur une population de survivants et sur leur descendance, ainsi que sur une population contrôle (parents et enfants, mêmes caractéristiques démographiques). \cite{yehuda2016} ont notamment montré que les niveaux de méthylation chez les survivants de l'Holocauste étaient corrélés aux niveaux de méthylation des enfants. Plus précisément, les gènes parentaux étaient caractérisés par un niveau de méthylation plus fort que les contrôles, alors que les gènes des descendants étaient caractérisés par un niveau de méthylation plus faible que les contrôles.\\
				
				Dans le même domaine, une autre étude toujours menée sur des survivants de l'Holocauste (\cite{yehuda2014}) s'intéresse à la relation entre le phénotype des parents (PTSD) et l'épigénétique de leurs enfants, nés après la période traumatique. Un autre gène très impliqué dans la régulation de l'axe HPA était au centre de l'étude: le gène \textbf{NR3C1} (et son promoteur au niveau de l'exon 1$_F$). L'expression de ce gène est directement liée à la synthèse des récepteurs aux glucocorticoïdes, et donc à la régulation du cortisol. Les auteurs ont mis en évidence un effet différentiel du PTSD maternel et paternel sur la méthylation du promoteur de NR3C1 chez les enfants. En effet, les niveaux de méthylation sont plus forts lorsque le père, mais pas la mère souffre de PTSD; alors que les niveaux de méthylation sont plus bas lorsque les deux parents sont touchés. Cela est le signe que la ``transmission'' épigénétique du PTSD paternel est modérée par le PTSD maternel; et par conséquent, c'est surtout le PTSD maternel qui détermine la susceptibilité au PTSD chez l'enfant.\\
				Par ailleurs, il a été montré que ces mesures de méthylation corrélaient avec une vulnérabilité des enfants aux troubles psychiatriques. Les enfants ayant au moins un parent affecté par le PTSD étaient plus sujets à des problèmes d'anxiété ou de dépression au cours de leur vie que les enfants sans parent PTSD. De plus, le PTSD maternel et le PTSD paternel ont donné lieu une fois de plus à des diagnostics différentiés. Le PTSD maternel était associé à l'anxiété et la dépression; alors que le PTSD paternel était associé à la présence de traumatisme dans l'enfance et plus de difficultés à moduler son ``style d'attachement''\footnote{cela est à mettre en relation avec la théorie de l'attachement chez l'enfant. Il existe 4 ``styles'' d'attachement: sécure, anxieux-soucieux, distant-évitant, craintif-évitant.}. Cette étude montre donc qu'au delà de l'épigénétique, les répercussions d'un traumatisme parental chez l'enfant peuvent bien être \textbf{phénotypiques}.
				\begin{figure}[H]
					\centering
					\includegraphics[width=300px]{../images/epi_hpa.png}
					\caption{Schéma récapitulatif des actions complexes des gènes NR3C1 et FKBP5 sur l'axe HPA dans le cas du PTSD (adapté de \cite{yehuda2013}, rouge=corrélation négative, vert=corrélation positive)}
				\end{figure}

		\subsubsection{Au niveau développemental}
			Beaucoup d'études s'intéressent au ``fetal programming'' dans un environnement adverse (en particulier, en présence de stress); mais peu se sont penchées sur le cas très précis des survivants de l'Holocauste. Cela est du au fait que, pour contrôler les influences de l'environnement post-natal, les études doivent être menées peu après (ou même pendant) la naissance, ce qui est désormais impossible dans le cas de l'Holocauste. Mais il est très probable que la Seconde Guerre mondiale, comme toute autre situation stressante, affecte le développement des enfants à naître. Dans une étude pilote, \cite{bercovich2014} ont ainsi comparé les statistiques sanitaires d'une cohorte de 70 Juifs nés dans des pays occupés entre 1940 et 1945, avec les statistiques d'une population contrôle. Ils ont conclu que le groupe d'individus exposés à la guerre \textit{in utero} avait notamment plus de risques de développer des maladies cardiovasculaires, de l'hypertension, des cancers. Ce groupe était aussi plus susceptible d'être touché par l'anxiété et la dépression. Cela dit, il est difficile de savoir si ces effets sont réellement dus à des facteurs développementaux ou plutôt à l'environnement de l'enfant après sa naissance. De plus, les troubles psychiatriques n'étaient pas le sujet central de l'étude.
			Cependant, cette étude montre que l'exposition à un traumatisme pendant le développement peut éventuellement laisser des traces à long terme, au niveau physique mais aussi au niveau psychologique.\\
			
			Pour compléter les résultats relativement peu développés de l'étude précédente, nous pouvons nous intéresser à un autre exemple d'épisode traumatique plus récent et donc plus propice à l'expérimentation : les attentats du 11 septembre 2001. Par exemple, \cite{brand2006} examinent les effets du stress maternel du aux attentats (PTSD, taux de cortisol) sur le développement de l'enfant \textit{in utero} et son comportement. Les femmes testées avaient été exposées directement aux attentats alors qu'elles étaient enceintes. Chez ces femmes, une corrélation négative entre le niveau de PTSD et les taux de cortisol a été décelée. Ce deux variables -- niveau de PTSD et cortisol -- étaient (respectivement) positivement et négativement corrélées avec le comportement de l'enfant après la naissance\footnote{comportement mesuré par l'inconfort de l'enfant en situation de contrainte, et son aversion pour la nouveauté}. Cependant, il est difficile dans ce genre d'étude de \textbf{dissocier} les facteurs épigénétiques des facteurs purement développementaux (cf. Figure \ref{fig:dev_gen}).
			\begin{figure}[H]
				\centering
				\includegraphics[width=300px]{../images/hpa_methyl.png}
				\caption{Liens entre développement et facteurs épigénétiques, \cite{vaiserman2017}}
				\label{fig:dev_gen}
			\end{figure}
	
		\subsubsection{Au niveau socio-historique}
			Au niveau social et familial, la transmission d'un traumatisme comme celui de l'Holocauste est particulièrement complexe. D'une part, les supports matériels (photographies, lettres...) manquent souvent; d'autre part, une loi implicite conduit les survivants à garder le \textbf{silence} sur les épreuves subies. Cette particularité du souvenir traumatique semble contrevenir à notre définition de la mémoire familiale, très axée sur la notion de transmission et la cohérence.\\
			
			Il est donc intéressant d'étudier la nature et la qualité de la communication entre les survivants de l'Holocauste (notés désormais SH) et leurs descendants. \cite{wiseman2006} se sont penchés sur les enfants de survivants (désormais ES), tandis que \cite{fossion2003} se sont davantage intéressés aux petit-enfants (désormais PES), en général moins étudiés. L'étude de \cite{fossion2003} portait plus précisément sur des familles dont les PES démontraient des troubles du comportement ou des troubles psychiatriques. Il a été constaté que les SH parlaient peu de leur expérience de l'Holocauste, mais étaient plus prompts à le faire avec PES. Avec les ES au contraire, la transmission du souvenir était largement inhibée, et il était même tacitement interdit de ``\textbf{méta-communiquer}'' à propos du silence\footnote{on parle parfois à ce sujet de ``conspiration du silence'', ou de ``double mur'' entre les SH qui ne racontent pas et ES qui n'interrogent pas.}. Ce comportement peut être expliqué par le fait que les SH, pour continuer à vivre, doivent oublier leur passé douloureux et se concentrer sur la transmission de la vie elle-même (au travers de leur famille) et non pas du traumatisme. La construction d'une famille devient ainsi une priorité et un dérivatif -- ce qui peut aussi conduire à des \textbf{comportement surprotecteurs} des SH envers les ES.\\
			
			Cela dit, les ES, même en l'absence de témoignage direct de la part de leur parents, se sentent responsable de leur bien-être et les voient comme vulnérables, ce qui entraîne notamment une \textbf{inversion des rôles familiaux} (les enfant deviennent parents de leurs parents). En particulier, les ES peuvent imaginer le passé de leur parents, et créer ainsi de nouveaux souvenirs éventuellement plus traumatiques encore que les souvenirs réels. La relation entre les SH et leurs enfants est donc très ambivalente, mais aussi très envahissante. Sur la base de récits produits par les ES\footnote{Relationship Anecdotes Paradigm (RAP), où le sujet doit raconter des anecdotes importantes à ses yeux et liées à ses relations avec son entourage.} mettant en jeu leurs relations avec leurs parents, \cite{wiseman2006} notent ainsi:
			\begin{itemize}
				\item une corrélation entre un sentiment de \textbf{colère} chez les ES et un comportement surprotecteur des SH -- ce qui tend à montrer que les ES souffrent d'un manque d'indépendance imposé par les SH;
				\item une corrélation entre un sentiment de \textbf{culpabilité} chez les ES\footnote{alors même qu'ils ne sont pas fautifs} et leur perception d'une vulnérabilité chez les SH -- ce qui tend à montrer que les ES ont intériorisé et tiennent compte du traumatisme de leurs parents;
				\item une absence de corrélation entre la colère des SH et la colère chez les ES -- ce tend à montrer que les ES ne réagissent pas à la colère pour ménager leurs parents.
			\end{itemize}\vspace{2mm}
			D'après \cite{fossion2003}, les ES sont plus sujets au PTSD bien sûr, mais aussi à des problèmes d'individuation, d'estime de soi, et des problèmes d'ordre relationnel (en particulier pour les relations intimes).\\
			
			Dans un tel contexte, les PES trouvent peu de place pour s'exprimer, et sont susceptibles de perdre leurs repères. Pour remédier à cela, \cite{fossion2003} ont proposé une thérapie basée sur la communication entre grand-parents et petit-enfant, pour rétablir la transmission de la mémoire familiale et la cohérence du groupe. Une dichotomie est établie entre ce qu'ils appellent les ``\textbf{souvenirs de vie}'' liés à l'Holocauste (en rapport avec la vie et la survie avant et après le traumatisme) et les ``souvenirs de mort'' (très négatifs, source du traumatisme). Les grand-parents sont ainsi libres de transmettre les souvenirs de vie, mais aussi libres de garder pour eux les souvenirs de mort. Avec cette thérapie \cite{fossion2003} permettent donc un renouveau du souvenir au niveau transgénérationnel en ``sautant une génération''. Des souvenirs inhibés par les grand-parents pour se protéger eux-mêmes et leurs enfants, aident 50 ans plus tard les petits-enfants à se resituer au sein de leur famille.
			\begin{figure}[H]
				\centering
				\includegraphics[width=400px]{../images/holocaust_family.png}
				\caption{Schéma récapitulatif des relations transgénérationnelles et de la circulation des souvenirs dans les familles marquées par l'Holocauste}
			\end{figure}
			
			
			
		
			
		\subsection{Critique générale et conclusion}
			La mémoire permet d'enregistrer des informations venant d'expériences et d'événements divers, de les conserver et de les restituer. Cette définition s'applique à nos trois ``types'' de mémoire, bien que de façon différentiée.
			\begin{table}[H]
				\centering
				\begin{tabular}{c|c|c|c}
					& Mémoire épigénétique & Mémoire foetale & Mémoire familiale \\ \hline
					Enregistrement & dans l'épigénome & \begin{minipage}{.25\textwidth}
						dans le cerveau en développement 
					\end{minipage}& \begin{minipage}{.25\textwidth}\vspace{2mm}
						dans l'histoire familiale:\\
						 -- consciences de ses membres\\
						 -- supports physiques\\
					\end{minipage} \\ \hline
					Conservation & par mitose et meïose & \begin{minipage}{.25\textwidth}
						par renforcement de l'apprentissage
					\end{minipage} & \begin{minipage}{.25\textwidth}\vspace{2mm}
						par transmission:\\
						-- aux nouveaux membres\\
						-- histoires, anecdotes, rituels\\
						par homogénéisation\\
						par aggrégation\vspace{2mm}
					\end{minipage} \\ \hline
					Restitution & \begin{minipage}{.25\textwidth}
						sur le phénotype ou dans le métabolisme
					\end{minipage}&  \begin{minipage}{.25\textwidth}\vspace{2mm}
						par répétition:\\
						-- mêmes comportements\\
						-- mêmes tendances
					\end{minipage}&  par ressassement 
				\end{tabular}
				\caption{Tableau récapitulatif des différents types de mémoire liés à la filiation, et de leur caractéristiques}
			\end{table}

			Les deux premiers types de mémoire que nous avons étudiés (mémoire épigénétique et mémoire fœtale) concernent la mémoire au sens très large: on parle ici plus d'une \textbf{conservation de prédispositions}, ou de préférences, d'une génération à l'autre.  Il est difficile d'aller à l'encontre de ces ``tendances'', dans la mesure où elles demeurent grandement \textbf{inconscientes}, et profondément ancrées (``imprimées'') en nous. Dans le cas du traumatisme par exemple, il est difficile de réguler de façon pharmacologique l'axe HPA lorsque les gènes liés à cet axe s'expriment mal. Par ailleurs, ces deux modes de transmission de la mémoire sont très liés entre eux, et certaines études ont pour but précis de les désintriquer\footnote{l'idée est par exemple de comparer les bébés portés par leur mère biologique aux bébés portés par une mère ayant bénéficié d'une FIV. Si les résultats sont les mêmes, alors les facteurs environnementaux priment sur les facteurs génétiques; sinon, ce sont les facteurs (épi)génétiques qui s'avèrent prépondérants (\cite{rice2009})}. Pour autant, cet héritage n'est sans doute \textbf{pas excessivement robuste} car il dépend aussi en grande partie de l'environnement. Si l'environnement (par exemple l'environnement familial) se montre plus clément pour les générations suivantes, les traces épigénétiques sont susceptibles de disparaître. De même, les préférences d'un individu acquises \textit{in utero} peuvent sans doute beaucoup changer au cours de la vie, en fonction de ce qu'il choisit d'expérimenter et des pressions qu'il subit\footnote{à ce sujet, il existe toute une littérature sur les liens entre exposition à des saveurs \textit{in utero} et neophobie}.\\
			
			D'un autre côté, la mémoire sociale liée à la famille se rapproche beaucoup plus de l'idée que l'on se fait de la mémoire, notamment à l'échelle de l'individu. La mémoire des membres d'une même famille n'est pas imperméable; elle dépend des souvenirs des autres et s'enrichit à leur contact. Dans le cadre de la mémoire traumatique, la présence au sein de la famille d'un individu atteint de troubles comme le PTSD peut altérer les \textbf{relations interpersonnelles} entre les membres, créer une atmosphère délétère de malaise et de non-dit. Contrairement à la mémoire de plus bas niveau, la mémoire sociale et familiale est hautement sémantisée et peut perdurer sur plusieurs générations; même si elle est sujette à de nombreuses variations. De plus, cette mémoire s'affranchit des ``liens de sang'', elle se construit à la faveur des interactions. Cela dit, une limite des études liées à la transmission de cette mémoire est qu'elles sont souvent très \textbf{qualitatives} et difficilement généralisables à un groupe où une classe sociale entière. Mais peut-être est-ce là aussi le propre de la mémoire d'une famille; ce qui lui permet de se dissocier des autres familles, comme le ferait un individu par rapport à ses pairs.\\
			
			Pour conclure, quels seraient les \textbf{leviers} permettant d'agir sur la transmission ou non de la mémoire? Il apparaît bien évidemment difficile d'agir au niveau (épi)génétique, et particulier sur l'humain. Pour autant, des métriques épigénétiques comme le taux de méthylation pourraient jouer le rôle de proxys (où \textit{biomarkers}) en vue de prévenir certaines affections, et de mesurer l'amélioration des symptômes liés à ces affections(\cite{yehuda2013}).\\
			Il est certainement plus facile de contrôler l'environnement pré-natal et post-natal, en assurant au fœtus des stimuli variés et en évitant les situations défavorables à son développement (stress, mauvaise alimentation, prise d'alcool etc.). Bien sûr, il n'est pas non plus nécessaire de tomber dans l'excès inverse\footnote{et de penser qu'un bébé à qui l'on lit du Kant deviendra grand philosophe, ou à qui l'on fait écouter du Mozart deviendra grand musicien!}, dans la mesure où les capacité cognitives du fœtus et ses facultés de remémoration demeurent limitées.\\
			À l'échelle du groupe enfin, la thérapie individuelle ou collective peut être une manière de faire resurgir des souvenirs de façon contrôlée (cf. \cite{fossion2003}) ou d'en faire oublier d'autres (cf. paradigme think-no think, \cite{anderson2001}). Dans le cas précis du PTSD, les thérapies cognitivo-comportementales sont les plus fiables et documentées\footnote{thérapie d'exposition, \textit{cognitive processing therapy}...}, mais d'autres thérapies davantage basées sur l'expression se sont aussi révélées efficaces\footnote{\textit{narrative exposure therapy}, \textit{brief eclectic psychotherapy}...}.
			
		%prévention
		%corrélation symptomes
	\newpage
	\defbibheading{maingeno}{\textbf{Génétique}}
	\defbibheading{maindevelo}{\textbf{Développement}}
	\defbibheading{mainsocio}{\textbf{Sciences sociales}}
	\defbibheading{minorgeno}{\textbf{Génétique}}
	\defbibheading{minordevelo}{\textbf{Développement}}
	\defbibheading{minorsocio}{\textbf{Sciences sociales}}
	\defbibheading{minorpsycho}{\textbf{Psychologie}}
	\section*{Références}
	\subsection*{Supports principaux}
	\printbibliography[heading=maingeno, keyword=main, keyword=geno]
	\printbibliography[heading=maindevelo, keyword=main, keyword=develo]
	\printbibliography[heading=mainsocio, keyword=main, keyword=socio]
	\subsection*{Supports secondaires}
	\printbibliography[heading=minorgeno, keyword=minor, keyword=geno]
	\printbibliography[heading=minordevelo, keyword=minor, keyword=develo]
	\printbibliography[heading=minorsocio, keyword=minor, keyword=socio]
	\printbibliography[heading=minorpsycho, keyword=minor, keyword=psycho]
	\nocite{*}

\end{document}
